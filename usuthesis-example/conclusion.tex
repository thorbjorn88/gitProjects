\chapter{CONCLUSION}

Doppler lidar can contribute useful data in determining altitude, speed, 
and position of a space craft during the landing phase of a mission. Under the
Autonomous Landing and Hazard Avoidance Technology (ALHAT) project, NASA has 
been developing Doppler lidar systems to provide this data \cite{amz12fiber}. It is clear that
Doppler lidar will play a role in forthcoming space exploration missions. 
The process of optimizing, improving performance, and developing novel scanning patterns
will be greatly aided by the availability of a simulation tool specific to Doppler lidar. 
Such a tool will be developed and added to the LadarSIM software package. 

Using the Doppler lidar simulation a trade-off study will be performed to 
determine the effects of laser transmission power, beam divergence,
aperture diameter, and FFT size on the performance of a Doppler lidar system.
The results of this research will simplify the task of designing or selecting
a Doppler lidar system by providing information about where to best allocate
resources to achieve performance requirements. 

Traditional and novel scanning patterns will be simulated and compared. This could
provide information on how a particular scanning pattern might improve the reliability
of data for a Doppler lidar. By simulating novel scanning patterns this research will 
explore the potential utility of Doppler lidar in terrain mapping and hazard avoidance
for a planetary landing mission. 



