\chapter{Introduction }
%%%%%%%% This line gets rid of page number on first page of text
\thispagestyle{empty}
%%%%%%%%%%%%%
\section{Overview}
Recent and upcoming missions for the exploration of solar system bodies require accurate
position and velocity data during the descent phase to ensure safe landing at pre-designated
sites. Because of inertial measurement unit (IMU) drift during travel, the data provided 
by the IMU may not be reliable. One solution proposed by NASA is the use of a frequency
modulated continuous wave (FMCW) Doppler lidar system during the landing phase to provide
additional information about attitude, position, and velocity to contribute to a successful
landing \cite{amz12,amz12fiber,amz12p2}. The optimization and comparison of potential configurations
of such a system would be greatly aided by an appropriate simulation tool.

LadarSIM is a robust parametrized simulation tool for time of flight lidar
developed at Utah State University's Center
for Advanced Imaging Ladar over the past decade and a half \cite{budgeLeishman,neilsenBudge}.
LadarSIM has the flexibility to simulate a wide range of time of flight
lidar systems with varying beam and scanner patterns as well as parametrized transmitters
and receivers.
These simulated systems can be evaluated using scenarios consisting of
user specified terrain, targets, and flight paths.

The research for this thesis had three objectives. First, to add a parameterized simulation
of an FMCW lidar transmitter and receiver, second, to utilize that simulation to perform
a trade-off study to determine the effects of varying laser transmission power, beam
divergence, aperture diameter, and FFT size on a Doppler lidar system, and third, to use the
simulation to compare traditional and novel scanning patterns in planetary landing 
scenarios. 

The simulation developed for this thesis is a useful tool for the development of Doppler lidar systems
with a wide range of transceiver characteristics and nearly arbitrary beam and 
scanning parameters. The information gained from the trade-off study will provide
valuable insight for someone designing or selecting a Doppler lidar system to meet
mission requirements. The point clouds generated in the exploration of novel scanning 
patterns give an idea of the potential utility of a Doppler lidar system for terrain mapping, navigation,
and hazard avoidance. 

\section{Chapter Outlines}
This thesis will describe the theory of FMCW lidar, how LadarSIM was expanded to add a simulation of Doppler lidar, 
and present results from LadarSIM's Doppler lidar simulation. To achieve this the thesis will be structured as 
follows. Chapter 2 will describe the theory behind FMCW radar detection and show how the same basic concepts can be 
applied to a lidar system. Chapter 3 will present some background information on LadarSIM and then show how it was 
modified to include a simulation of Doppler lidar. Chapter 4 will present a trade-off study exploring the effects 
of laser transmission power, beam divergence, aperture diameter, and FFT size on a Doppler lidar system. Chapter 5
will discuss novel scanning patterns and present point clouds generated by LadarSIM for those patterns. Chapter 6 
is the conclusion of the thesis and will also discuss the potential for future work. 




