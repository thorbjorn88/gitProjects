%
%  This is an example of how a LaTeX thesis should be formatted.  This
%  document contains chapter 1 of the thesis.
%
%  Time-stamp: "[sample-chapter1.tex] last modified by Scott Budge (scott) on 2016-07-27 (Wednesday, 27 July 2016) at 17:04:08 on goga.ece.usu.edu"
%
%  Info: $Id: sample-chapter1.tex 967 2016-07-28 15:33:29Z scott $   USU
%  Revision: $Rev: 967 $
% $LastChangedDate: 2016-07-28 09:33:29 -0600 (Thu, 28 Jul 2016) $
% $LastChangedBy: scott $
%

\chapter{LITERATURE REVIEW}

Adany et al.
describe the operation and performance of a simplified homodyne detection
scheme.
In this scheme a waveform generator drives an electro-optic modulator which
modulates an optical signal from a laser.
The resulting signal is split into two parts.
One part is amplified and sent to the telescope and the other is used as
the local oscillator.
The returning signal from the telescope is optically mixed with the local
oscillator signal and detected by a balanced photodetector.
An FFT is then performed on the detected signal to determine detected range
and speed.
\cite{adany09}.
This scheme is on NASA's Morphious test vehicle in order to demonstrate
it's utility as a potential instrument for planetary landing missions.
\cite{amz12}.
Because of its use by NASA the simplified homodyne detection scheme has
been selected as the basis of the doppler lidar simulation in LadarSIM.

Adany et al.
tested their lidar at ground level at ranges of 50 m and 370 m \cite{adany09}.
The NASA instrument has been tested mounted on a helicopter and the morpheus
test vehicle to demonstrate its performance.
In the altitude tests the altidue varried from approximately 50 m to 1700
m (pierrottet papers).
The Morpheus test vehicle was launched to a 250 m altitude then descended
along a 30 degree path to a simulated landing field.
These tests provide useful data to compare with simulated results.
The results of Adany et al.
are particularly interesting because they include a return spectrum from
the 370m experiment.
The experiments of Adany et al.
fall short in that the targets all had relatively good reflectivity and
an approximately 90 degree angle incidence therefore there is little to
be learn about spectra resulting from non-ideal conditions.
The expiriments performed by NASA accurately mimic real world circumstances
but no data on the resulting spectra is provided and no information on
the performance of the system at long ranges.The simulation proposed will
therefore provide insight on data which is not readily available, performance
information and spectra from non-ideal scenarios.
This data will be useful in making decisions about doppler lidar system
characteristics as well as improving detection algorithms for doppler lidar.
