%
%  This is an example of how a LaTeX thesis should be formatted.  This
%  document contains chapter 1 of the thesis.
%
%  Time-stamp: "[sample-chapter1.tex] last modified by Scott Budge (scott) on 2016-07-27 (Wednesday, 27 July 2016) at 17:04:08 on goga.ece.usu.edu"
%
%  Info: $Id: sample-chapter1.tex 967 2016-07-28 15:33:29Z scott $   USU
%  Revision: $Rev: 967 $
% $LastChangedDate: 2016-07-28 09:33:29 -0600 (Thu, 28 Jul 2016) $
% $LastChangedBy: scott $
%

\chapter{LITERATURE REVIEW}

\section{Frequency Modulated Continuous Waveform Homodyne Detection}
A continuous wave radar in which a single microwave oscillator serves as both
the transmitter and local oscillator (LO) is, generally speaking, a homodyne radar.
Frequency modulated continuous waveform (FMCW) radar systems often leverage a 
homodyne architecture. In FMCW homodyne radar, the continuous wave signal is modulated
to create a linear chirp which is transmitted via antenna toward a target. The return
echo signal, which is delayed in time, is mixed with the LO signal. The result is a signal 
which is comprised of a linearly increasing chirp signal, which is actively filtered out, and 
the beat frequency which is used for detection \cite{brooker2009}. 

Because microwaves and lasers are both electromagnetic radiation, just in different
segments of the electromagnetic spectrum, an FMCW homodyne lidar can be developed by 
replacing components such as the antenna and the microwave oscillator with equivalent 
devices for creating, transmitting, receiving, and analyzing a laser signal. 


\section{Simplified Homodyne Detection}
Adany et al.
describe the operation and performance of a simplified homodyne detection
scheme.
In this scheme a waveform generator drives an electro-optic modulator which
modulates an optical signal from a laser.
The resulting signal is split into two parts.
One part is amplified and sent to the telescope and the other is used as
the local oscillator.
The returning signal from the telescope is optically mixed with the local
oscillator signal and detected by a balanced photodetector.
An FFT is then performed on the detected signal to determine detected range
and speed.
\cite{adany09}.
This scheme was tested on NASA's Morpheus test vehicle in order to demonstrate
its utility as a potential instrument for planetary landing missions.
\cite{amz12,amz12p2,amz12fiber,amz16coherent}.
Because of its use by NASA, the simplified homodyne detection scheme has
been selected as the basis of the Doppler lidar simulation in LadarSIM.

\section{Simplified Homodyne Performance}
Adany et al.
tested their lidar at ground level at ranges of 50 m and 370 m \cite{adany09}.
The NASA instrument has been tested mounted on a helicopter and the morpheus
test vehicle to demonstrate its performance.
In the helicopter tests, the altitude was varied from approximately 50 m to 1700
m \cite{pierrottet2009flight,pierrottet2011navigation}.
The Morpheus test vehicle was launched to a 250 m altitude then descended
along a 30 degree path to a simulated landing field \cite{amz16coherent}.
These tests provide useful data to compare with simulated results.
The results of Adany et al.
are particularly interesting because they include a return spectrum from
the 370m experiment which will be useful in comparing to simulated spectra.
The experiments of Adany et al.
fall short in that the targets all had relatively good reflectivity and
an approximately 90 degree angle incidence; therefore, there is little to
be learned about spectra resulting from non-ideal conditions.
The experiments performed by NASA accurately mimic real world circumstances
but no data on the resulting spectra is provided and no information on
the performance of the system at long ranges.The simulation proposed will
therefore provide insight on data which is not readily available, performance
information and spectra from non-ideal scenarios.
This data will be useful in making decisions about Doppler lidar system
characteristics as well as improving detection algorithms for Doppler lidar.

\section{Spectrum Analysis Methods}
A key step in the detection of FMCW lidar or radar is obtaining the frequency spectrum 
of the returned signal. This is typically done with an FFT. Resolution of range and speed
is inextricably tied to the resolution in frequency domain. Hence, obtaining many samples
of the returned signal and performing a large FFT on that signal allows for good 
resolution. While it is necessary that the FFT be large, only a relatively small 
number of the bins will contain peaks of the return signal. In order to reduce computational
complexity and potentially increase performance, alternative methods can be used to obtain 
the return spectrum. Al-Qudsi et al. demonstrate 
how a Zoom FFT (ZFFT) can be used to significantly decrease the computational complexity
fo FMCW radar. The ZFFT works by determining what portion of the total signal
is of interest then filtering the signal to isolate the signal of interest so that a smaller
FFT can be performed \cite{Al-QudsiZoomfft}. A second approach is the Chirp Z-transform. In short, 
the Chirp Z-transform differs from the FFT in that the spacing of frequency bins does not need
to be uniform. Using the Chirp Z-transform frequency bins can be placed densely where high
resolution is required and sparsely where it is not, potentially resulting in better resolution
on important details with less computational complexity \cite{Pen-ChengChirpZ,wangThresh}. 

\section{Probability of Dropout and False Alarm}
Two very important metrics in evaluating the performance of a Doppler lidar system are probability
of detection and false alarm rate. Probability of detection describes the probability that a detection
will not be missed and the probability of false alarm describes the probability that something will be
detected which is not part of the true return signal. Both of these phenomena are cause by noise in the signal
either pulling a returned signal below the detection threshold or simply noise peaking above the detection 
threshold. In order to simulate this behavior, it is necessary to simulate the noise in the bins of the FFT. Mark A. Richards
provides an in-depth derivation of this \cite{richards2007dftnoise}. Wang et al. also examine the effects
of noise on an FMCW radar's performance along with picket fence effect and derive the detection probability
and false alarm probability as well as propose an improved threshold detection method \cite{wangThresh}. 