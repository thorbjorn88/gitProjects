%
%  This is an example of how a LaTeX thesis should be formatted.  This
%  document contains chapter 1 of the thesis.
%
%  Time-stamp: "[sample-chapter1.tex] last modified by Scott Budge (scott) on 2016-07-27 (Wednesday, 27 July 2016) at 17:04:08 on goga.ece.usu.edu"
%
%  Info: $Id: sample-chapter1.tex 967 2016-07-28 15:33:29Z scott $   USU
%  Revision: $Rev: 967 $
% $LastChangedDate: 2016-07-28 09:33:29 -0600 (Thu, 28 Jul 2016) $
% $LastChangedBy: scott $
%

\chapter{INTRODUCTION}
%%%%%%%% This line gets rid of page number on first page of text
\thispagestyle{empty}
%%%%%%%%%%%%%

LadarSIM is a robust parametrized simulation tool for time of flight lidar
developed at Utah State University's Center
for Advanced Imaging Ladar over the past decade and a half \cite{budgeLeishman,neilsenBudge}.
LadarSIM has the flexibility to simulate a wide range of time of flight
lidar systems with varying beam and scanner patterns and parametrized transmitt
ers and receivers.
These simulated systems can be evaluated using scenarios consisting of
user specified terrain, targets, and flight paths.
Taking full advantage of the significant work that has gone into creating
the LadarSIM software, a simulation of a parametrized FMCW Doppler transmitter
and receiver will be added to LadarSIM to evaluate the performance of Doppler
lidar systems.
Utilizing this Doppler lidar simulation capability a trade-off study will
be performed to determine the effects of varying laser transmission power,
beam divergence, aperture diameter, and FFT size on a Doppler lidar system.
These parameters have been selected because of the significant effect they
can have on the performance, cost, size, and power consumption of a Doppler
lidar system.

It is anticipated that the results of this research will present a clear
demonstration of the effects of these parameters on a Doppler lidar system.
For each of these parameters it is expected that performance will increase
as the value of the parameter is improved but diminishing returns will
be evident.
Using this data, a designer would have strong guidance in designing the
parameters of a Doppler lidar system to meet mission requirements while
expending the minimum necessary resources.

