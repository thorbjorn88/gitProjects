%
%  This is an example of how a LaTeX thesis should be formatted.  This
%  document contains chapter 1 of the thesis.
%
%  Time-stamp: "[sample-chapter1.tex] last modified by Scott Budge (scott) on 2016-07-27 (Wednesday, 27 July 2016) at 17:04:08 on goga.ece.usu.edu"
%
%  Info: $Id: sample-chapter1.tex 967 2016-07-28 15:33:29Z scott $   USU
%  Revision: $Rev: 967 $
% $LastChangedDate: 2016-07-28 09:33:29 -0600 (Thu, 28 Jul 2016) $
% $LastChangedBy: scott $
%

\chapter{INTRODUCTION}
%%%%%%%% This line gets rid of page number on first page of text
\thispagestyle{empty}
%%%%%%%%%%%%%
Recent and upcoming missions for the exploration of solar system bodies require accurate
position and velocity data during the descent phase ensure safe landing at pre-designated
sites. Because of inertial measurement unit (IMU) drift during travel, the data provided 
by the IMU may not be reliable. One solution proposed by NASA is the use of a frequency
modulated continuous wave (FMCW) Doppler lidar system during the landing phase to provide
additional information about attitude, position, and velocity to contribute to a successful
landing \cite{amz12,amz12fiber,amz12p2}. The optimization and comparison of potential configurations
of such a system would be greatly aided by an appropriate simulation tool.

LadarSIM is a robust parametrized simulation tool for time of flight lidar
developed at Utah State University's Center
for Advanced Imaging Ladar over the past decade and a half \cite{budgeLeishman,neilsenBudge}.
LadarSIM has the flexibility to simulate a wide range of time of flight
lidar systems with varying beam and scanner patterns as well as parametrized transmitters
and receivers.
These simulated systems can be evaluated using scenarios consisting of
user specified terrain, targets, and flight paths.

The proposed research has three objectives. First to add a parameterized simulation
of an FMCW lidar transmitter and receiver, second utilize that simulation to perform
a trade-off study to determine the effects of varying laser transmission power, beam
divergence, aperture diameter, and FFT size on a Doppler lidar system, and to use the
simulation to compare traditional and novel scanning patterns in planetary landing 
scenarios. 

The resulting simulation will be useful in the development of Doppler lidar systems
with a wide range of transceiver characteristics and nearly arbitrary beam and 
scanning parameters. The information gained from the trade-off study will provide
valuable insight for someone designing or selecting a Doppler lidar system to meet
mission requirements. Point clouds from the comparison of scanning patterns will 
test the potential utility of a Doppler lidar system for terrain mapping, navigation,
and hazard avoidance. 

%Utilizing this Doppler lidar simulation capability a trade-off study will
%be performed to determine the effects of varying laser transmission power,
%beam divergence, aperture diameter, and FFT size on a Doppler lidar system.
%These parameters have been selected because of the significant effect they
%can have on the performance, cost, size, and power consumption of a Doppler
%lidar system.

%It is anticipated that the results of this research will present a clear
%demonstration of the effects of these parameters on a Doppler lidar system.
%For each of these parameters it is expected that performance will increase
%as the value of the parameter is improved but diminishing returns will
%be evident.
%Using this data, a designer would have strong guidance in designing the
%parameters of a Doppler lidar system to meet mission requirements while
%expending the minimum necessary resources.

