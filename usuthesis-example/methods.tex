%
%  This is an example of how a LaTeX thesis should be formatted.  This
%  document contains chapter 1 of the thesis.
%
%  Time-stamp: "[sample-chapter1.tex] last modified by Scott Budge (scott) on 2016-07-27 (Wednesday, 27 July 2016) at 17:04:08 on goga.ece.usu.edu"
%
%  Info: $Id: sample-chapter1.tex 967 2016-07-28 15:33:29Z scott $   USU
%  Revision: $Rev: 967 $
% $LastChangedDate: 2016-07-28 09:33:29 -0600 (Thu, 28 Jul 2016) $
% $LastChangedBy: scott $
%

\chapter{RESEARCH AND DESIGN METHODS}
The goal of this research project is to ascertain and demonstrate the effects
of particular parameters on the performance of a Doppler lidar system.
The parameters to be studied are laser transmission power, beam divergence,
aperture diameter, and FFT size.
These parameters have been selected because of their potential effect on
the overall performance, cost, power consumption, weight, and computational
power of a system.
The results of this research will simplify the task of designing or selecting
a Doppler lidar system by providing information about where to best allocate
resources to achieve performance requirements.

In order to simulate the effects of these characteristics on a Doppler lidar
system, a detailed parametrized simulation of Doppler lidar will be added
to the LadarSIM lidar simulation software.
The updated LadarSIM software will then be used to simulate various Doppler
lidar systems with varying transmission power, beam divergence, aperture
diameter, and FFT size then the performance of these instruments will be
evaluated in a variety of simulated scenarios by calculating and recording
the probabilities of detection and false alarm.
LadarSIM is a robust and realistic simulator which accurately simulates
the real world behavior of a lidar system.
After simulating the true measurements of range and velocity of a scenario,
LadarSIM simulates the transmission and return of a beam by modeling the
interactions of the beam with the atmosphere, target, receiver optics,
and electronics to obtain a simulated returned signal.
Using this returned signal and models of the noise in the system the probability
of detection and false alarm are calculated.
For each simulated beam, the probability of detection and false alarm will
be recorded with information about the range and angle of incident to the
target.
This data will be used to compare the performance of Doppler lidar systems
with different parameters.
The information about the range and angle of incident will be used to compare
performance under different conditions.

In order to make the effect of each parameter as evident as possible, each
parameter will be varied and measured while holding the other parameters
constant.
This will also cut down on the number of simulations because it will not
be necessary to simulate every possible combination of parameters to determine
the effects of each parameter.
However, the parameters of aperture diameter and laser transmission power
are directly related.
An increase in one will proportionally offset a decrease in the other.
This relationship will be studied and the resulting trade-offs presented.