\chapter{Frequency Modulated Continuous Wave Detection}

\section{FMCW Basics}
The theory behind Frequency Modulated Continuous Wave (FMCW) detection will 
now be explored following the development of Brooker\cite{brooker2009}.
FMCW detection refers to a radar/lidar 
system in which a continuous wave of known frequency is modulated in amplitude,
transmitted, and the reflected signal is detected. A continuous wave radar in 
which a single microwave oscillator serves as both
the transmitter and local oscillator (LO) is, generally speaking, a homodyne radar.
Frequency modulated continuous waveform (FMCW) radar systems often leverage a 
homodyne architecture \cite{brooker2009}.

An FMCW radar uses a continuous wave signal which is modulated in amplitude 
over a range of frequencies creating a linear chirp. This chirped signal is 
radiated to the target and an echo 
returns after time $T_p$, which is the time it takes for the signal to reach the
target and reflected energy to return to the antennae. The returned signal is mixed
with the signal from the LO producing a beat signal at frequency $f_b$.  Figure~\ref{fig:doppShift}
illustrates this process. 

\begin{figure}
	\begin{center}
		\begin{tikzpicture}[every text node part/.style={align=center}]
			\draw[thick,->] (1,1) -- (7,1) node[anchor=north] {Time};
			\draw[thick,->] (1,1) -- (1,5) node[anchor=south] {Freq};
			\draw (1,1) -- (5,5);
			\draw[dashed] (2.5,1) -- (6.5,5);
			\draw[dashed,<->] (4,2.6) -- (4,3.9) node[pos=0.5, anchor = east] {fb};
			\draw[dashed,<->] (1,.8) -- (2.5,.8) node[pos=0.5, anchor = north] {Tp};
			\draw[dashed,<->] (1,0) -- (5,0) node[pos=0.5, anchor = north] {Tb};
		\end{tikzpicture}
		\caption{Transmit and Receive Doppler Shift}
		\label{fig:doppShift}
	\end{center}
\end{figure}



A continuous wave radar in which a single microwave oscillator serves as both
the transmitter and local oscillator (LO) is, generally speaking, a homodyne radar.
Frequency modulated continuous waveform (FMCW) radar systems often leverage a 
homodyne architecture. In FMCW homodyne radar the continuous wave signal is modulated
to create a linear chirp which is transmitted via antenna toward a target. The return
echo signal, which is delayed in time, is mixed with the LO signal. The result is a signal 
which is comprised of a linearly increasing chirp signal, which is actively filtered out, and 
the beat frequency which is used for detection \cite{brooker2009}. 

\section{Simplified Homodyne Detection}