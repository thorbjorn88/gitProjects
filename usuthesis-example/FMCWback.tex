\chapter{Frequency Modulated Continuous Wave Detection}

\section{FMCW Basics}
Frequency Modulated Continuous Wave (FMCW) detection refers to a radar/lidar 
system in which a continuous wave of known frequency is modulated in amplitude,
transmitted, and the reflected signal is detected. A continuous wave radar in 
which a single microwave oscillator serves as both
the transmitter and local oscillator (LO) is, generally speaking, a homodyne radar.
Frequency modulated continuous waveform (FMCW) radar systems often leverage a 
homodyne architecture.




A continuous wave radar in which a single microwave oscillator serves as both
the transmitter and local oscillator (LO) is, generally speaking, a homodyne radar.
Frequency modulated continuous waveform (FMCW) radar systems often leverage a 
homodyne architecture. In FMCW homodyne radar the continuous wave signal is modulated
to create a linear chirp which is transmitted via antenna toward a target. The return
echo signal, which is delayed in time, is mixed with the LO signal. The result is a signal 
which is comprised of a linearly increasing chirp signal, which is actively filtered out, and 
the beat frequency which is used for detection \cite{brooker2009}. 

\section{Simplified Homodyne Detection}